\documentclass{article}
\usepackage[utf8]{inputenc}
\usepackage{hyperref}
\usepackage{listings}
\usepackage{xcolor}
\usepackage{geometry}
\usepackage[T1]{fontenc}
\geometry{a4paper, margin=1in}

\definecolor{codegray}{rgb}{0.5,0.5,0.5}
\definecolor{codeblue}{rgb}{0.13,0.13,1}
\definecolor{codegreen}{rgb}{0,0.6,0}

\lstdefinestyle{mystyle}{
    backgroundcolor=\color{white},
    commentstyle=\color{codegray},
    keywordstyle=\color{codeblue},
    numberstyle=\tiny\color{codegray},
    stringstyle=\color{codegreen},
    basicstyle=\ttfamily\footnotesize,
    breakatwhitespace=false,
    breaklines=true,
    captionpos=b,
    keepspaces=true,
    numbers=left,
    numbersep=5pt,
    showspaces=false,
    showstringspaces=false,
    showtabs=false,
    tabsize=2
}
\lstset{style=mystyle}

\title{Documentação do Sistema de Cadastro de Usuários no Laravel}
\author{Bruno}
\date{\today}

\begin{document}

\maketitle

\section{Introdução}
Este documento detalha o desenvolvimento de um sistema de cadastro de usuários no Laravel. O projeto envolve a criação de uma migration personalizada, um controller para manipulação dos dados, views para exibição do formulário e página de sucesso, além da configuração das rotas.

\section{Criando a Migration Personalizada}
O primeiro passo foi criar uma migration para a tabela personalizada \texttt{usuarios}. Para isso, utilizei o comando:

\begin{lstlisting}[language=bash]
php artisan make:migration create_usuarios_table
\end{lstlisting}

Em seguida, editei o arquivo gerado e defini a estrutura da tabela:

\begin{lstlisting}[language=PHP]
use Illuminate\Database\Migrations\Migration;
use Illuminate\Database\Schema\Blueprint;
use Illuminate\Support\Facades\Schema;

return new class extends Migration {
    public function up(): void {
        Schema::create('usuarios', function (Blueprint $table) {
            $table->id();
            $table->string('nome', 100);
            $table->string('email')->unique();
            $table->string('senha');
            $table->string('telefone', 15)->nullable();
            $table->text('endereco')->nullable();
            $table->date('data_nascimento')->nullable();
            $table->boolean('ativo')->default(true);
            $table->timestamps();
        });
    }
    public function down(): void {
        Schema::dropIfExists('usuarios');
    }
};
\end{lstlisting}

Depois, rodei a migration:

\begin{lstlisting}[language=bash]
php artisan migrate
\end{lstlisting}

\section{Criando o Model}
Para facilitar a manipulação dos usuários no banco:

\begin{lstlisting}[language=bash]
php artisan make:model Usuario
\end{lstlisting}

Editei \texttt{app/Models/Usuario.php}:

\begin{lstlisting}[language=PHP]
namespace App\Models;

use Illuminate\Database\Eloquent\Model;

class Usuario extends Model {
    protected $table = 'usuarios';
    protected $fillable = ['nome', 'email', 'senha', 'telefone', 'endereco', 'data_nascimento', 'ativo'];
    protected $hidden = ['senha'];
}
\end{lstlisting}

\section{Criando o Controller}

Criei um controller:

\begin{lstlisting}[language=bash]
php artisan make:controller UsuarioController
\end{lstlisting}

Implementei os métodos:\texttt{create} e \texttt{store}:

\begin{lstlisting}[language=PHP]
namespace App\Http\Controllers;

use App\Models\Usuario;
use Illuminate\Http\Request;
use Illuminate\Support\Facades\Hash;

class UsuarioController extends Controller {
    public function create() {
        return view('cadastro');
    }
    public function store(Request $request) {
        $request->validate([
            'nome' => 'required|string|max:100',
            'email' => 'required|email|unique:usuarios,email',
            'senha' => 'required|min:6'
        ]);
        Usuario::create([
            'nome' => $request->nome,
            'email' => $request->email,
            'senha' => Hash::make($request->senha)
        ]);
        return view('cadastro');
    }
}
\end{lstlisting}

\section{Criando as Rotas}

Adicionei as rotas em \texttt{routes/web.php}:

\begin{lstlisting}[language=PHP]
use App\Http\Controllers\UsuarioController;
use Illuminate\Support\Facades\Route;

Route::get('/cadastro', [UsuarioController::class, 'create'])->name('cadastro');
Route::post('/cadastro', [UsuarioController::class, 'store'])->name('cadastro.store');
\end{lstlisting}

\section{Criando as Views}

\subsection{Formulário de Cadastro}
Criei a view \texttt{resources/views/cadastro.blade.php}:

\begin{lstlisting}[language=HTML]
<form action="{{ route('cadastro.store') }}" method="POST">
    @csrf
    <label>Nome:</label>
    <input type="text" name="nome" required>
    <label>Email:</label>
    <input type="email" name="email" required>
    <label>Senha:</label>
    <input type="password" name="senha" required>
    <button type="submit">Cadastrar</button>
</form>
\end{lstlisting}

\subsection{Tela de Sucesso}
Criei \texttt{resources/views/cadastro\_sucesso.blade.php}:

\begin{lstlisting}[language=HTML]
<h2>Cadastro realizado com sucesso!</h2>
<a href="/">Voltar para a p\'agina inicial</a>
\end{lstlisting}

\section{Conclusão}
Este sistema foi construído utilizando Laravel, com um fluxo completo de cadastro de usuários, incluindo validação de dados e exibição de mensagens ao usuário. 

\end{document}


